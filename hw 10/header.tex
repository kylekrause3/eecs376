\RequirePackage{microtype}
\RequirePackage{mathtools}
\RequirePackage{amsthm}
\RequirePackage{amssymb}
\RequirePackage{xspace}
\RequirePackage[shortlabels]{enumitem}
\RequirePackage{xcolor}
\RequirePackage{hyperref}
\RequirePackage[capitalize,nameinlink,noabbrev]{cleveref} % must load after hyperref
\usepackage{fvextra} % To use Verbatim
\RequirePackage[boxed]{algorithm}
\RequirePackage[noend]{algpseudocode}
\RequirePackage{tikz}

\hypersetup{breaklinks=true,
    colorlinks=true,
    linkcolor=blue,
    filecolor=blue,
    citecolor=blue,
    urlcolor=blue}

\algrenewcommand{\algorithmiccomment}[1]{\texttt{//} #1}
\algrenewcommand\algorithmicrequire{\textbf{Input:}}
\algrenewcommand\algorithmicensure{\textbf{Output:}}


% allow cleveref to label and reference enumerables defined in the exam class.
% these automatically define corresponding \Crefname as well.

% https://tex.stackexchange.com/questions/126020/cleveref-doesnt-use-correct-capitalized-name-if-used-with-amsthm
\makeatletter
\if@cref@capitalise
\crefname{question}{Question}{Questions}
\Crefname{partno}{Part}{Parts}
\crefname{subpart}{Subpart}{Subparts)}
\crefname{subsubpart}{Subsubpart}{Subsubparts}
\else
\crefname{question}{question}{questions}
\crefname{partno}{part}{parts}
\crefname{subpart}{subpart}{subparts}
\crefname{subsubpart}{subsubpart}{subsubparts}
\fi
\makeatother

% numeric sets in "blackboard" font

\newcommand{\N}{\mathbb{N}}
\newcommand{\Z}{\mathbb{Z}}
\newcommand{\R}{\mathbb{R}}
\newcommand{\Q}{\mathbb{Q}}
\newcommand{\C}{\mathbb{C}}
\newcommand{\Prop}{\mathbb{P}}

% paired delimiters

\DeclarePairedDelimiter\abs{\lvert}{\rvert}
\DeclarePairedDelimiter\length{\lVert}{\rVert}
\DeclarePairedDelimiter\norm{\lVert}{\rVert}
\DeclarePairedDelimiter\parens{(}{)}
\DeclarePairedDelimiter\tuple{(}{)}
\DeclarePairedDelimiter\brackets{[}{]}
\DeclarePairedDelimiter\floor{\lfloor}{\rfloor}
\DeclarePairedDelimiter\ceil{\lceil}{\rceil}
\DeclarePairedDelimiter\round{\lfloor}{\rceil}
\DeclarePairedDelimiter\set{\{}{\}}
\DeclarePairedDelimiter\inner{\langle}{\rangle}

\newcommand{\bit}{\set{0,1}}

% asymptotics

\DeclareMathOperator{\Otil}{\tilde{O}}
\DeclareMathOperator{\poly}{poly}
\DeclareMathOperator{\polylog}{polylog}
\DeclareMathOperator{\negl}{negl}

% algorithms

\newcommand{\algo}[1]{\textsc{#1}}

\newcommand{\memo}{\text{memo}}
\newcommand{\tabl}{\text{table}}
\newcommand{\backtrack}{\text{backtrack}}

\newcommand{\ALG}{\text{ALG}}
\newcommand{\OPT}{\text{OPT}}
\newcommand{\weight}{\text{weight}}
\newcommand{\val}{\text{value}}

% computability

% named language
\newcommand{\lang}[1]{L_{\text{#1}}}
% computational problem
\newcommand{\cproblem}[1]{\ensuremath{\text{#1}}\xspace}
% class of languages
\newcommand{\class}[1]{\ensuremath{\mathsf{#1}}\xspace}

\newcommand{\qst}{q_{\text{start}}}
\newcommand{\qacc}{q_{\text{acc}}}
\newcommand{\qrej}{q_{\text{rej}}}

\newcommand{\Lbarber}{\lang{BARBER}}
\newcommand{\atm}{\lang{ACC}}
\newcommand{\htm}{\lang{HALT}}
\newcommand{\ehtm}{\lang{$\varepsilon$-HALT}}
\newcommand{\eqtm}{\lang{EQ}}
\newcommand{\etm}{\lang{$\emptyset$}}
\newcommand{\epstm}{\lang{$\set{\varepsilon}$}}
\newcommand{\Lprop}{\lang{$\Prop$}}
\newcommand{\LSigmastar}{\lang{$\Sigma^*$}}

% complexity

\newcommand{\yes}{\ensuremath{\text{YES}}}
\newcommand{\no}{\ensuremath{\text{NO}}}

\newcommand{\DTIME}{\class{DTIME}}
\renewcommand{\P}{\class{P}}
\newcommand{\NP}{\class{NP}}
\newcommand{\NPH}{\class{NPH}}
\newcommand{\NPC}{\class{NPC}}
\newcommand{\coNP}{\class{coNP}}

\newcommand{\MAZE}{\cproblem{MAZE}}
\newcommand{\PALINDROME}{\cproblem{PALINDROME}}
\newcommand{\TSP}{\cproblem{TSP}}
\newcommand{\SAT}{\cproblem{SAT}}
\newcommand{\CSAT}{\cproblem{CSAT}}
\newcommand{\TSAT}{\cproblem{3SAT}}
\newcommand{\VC}{\cproblem{VERTEX-COVER}}
\newcommand{\SC}{\cproblem{SET-COVER}}
\newcommand{\HC}{\cproblem{HAMCYCLE}}
\newcommand{\HP}{\cproblem{HAMPATH}}
\newcommand{\IS}{\cproblem{IS}}
\newcommand{\CLIQUE}{\cproblem{CLIQUE}}
\newcommand{\SSUM}{\cproblem{SUBSET-SUM}}
\newcommand{\KNAPSACK}{\cproblem{KNAPSACK}}
\newcommand{\MAXCUT}{\cproblem{MAX-CUT}}

% randomness

\DeclareMathOperator*{\Var}{Var}
\DeclareMathOperator*{\Ex}{\mathbb{E}}

\newcommand{\RP}{\class{RP}}
\newcommand{\coRP}{\class{coRP}}
\newcommand{\BPP}{\class{BPP}}
\newcommand{\ZPP}{\class{ZPP}}
\newcommand{\BQP}{\class{BQP}}

%%% misc

\newcommand{\eps}{\varepsilon}

%%% theorems

\theoremstyle{plain}            % following are "theorem" style

\newtheorem{theorem}{Theorem}
\newtheorem{lemma}[theorem]{Lemma}
\newtheorem{corollary}[theorem]{Corollary}
\newtheorem{proposition}[theorem]{Proposition}
\newtheorem{claim}[theorem]{Claim}
\newtheorem{fact}[theorem]{Fact}
\newtheorem{openproblem}[theorem]{Open Problem}

\theoremstyle{definition}       % following are def style

\newtheorem{definition}[theorem]{Definition}
\newtheorem{conjecture}[theorem]{Conjecture}
\newtheorem{protocol}[theorem]{Protocol}
\newtheorem{exercise}[theorem]{Exercise}

\theoremstyle{remark}           % following are remark style

\newtheorem{example}[theorem]{Example}
\newtheorem{remark}[theorem]{Remark}
\newtheorem{note}[theorem]{Note}

%%% for homework and section notes

\newcommand{\commonheader}[2]{
    \pagestyle{headandfoot}
    \setlength{\headheight}{26pt}
    \setlength{\headsep}{16pt}

    \header
        {\small{\textbf{EECS 376: Foundations of Computer Science}} \\ \footnotesize{\textbf{University of Michigan, Winter 2025}}}
        {#1}
        {#2}

    \firstpageheadrule
    \runningheadrule

    \footer
        {}
        {\thepage}
        {}
}

\newcommand{\hwheader}{
    \commonheader
        {\Large \textbf{Homework \hwnum}}
        {\small \textbf{Due 8:00pm, \duedate\\ {\tiny(accepted until 9:59 pm, no credit after)}}}
}

\newcommand{\hwslnheader}{
    \commonheader
    	{}
        {\Large \textbf{Solutions to Homework \hwnum}}
    \printanswers
}

\newcommand{\notesheader}{
    \commonheader
    	{}
        {\Large \textbf{Discussion Notes \sectionnum}}
}

\newcommand{\practiceheader}{
    \commonheader
    	{}
        {\Large \textbf{Discussion Worksheet \sectionnum}}
}

\newcommand{\practiceslnheader}{
    \commonheader
    	{}
        {\Large \textbf{Solutions to Discussion Worksheet \sectionnum}}
}

\newcommand{\reviewheader}{
    \commonheader 
    \smallskip
    	{}
        {\Large \textbf{Midterm Review Notes}}
}

\newcommand{\hwpreface}{

\noindent This homework has \numquestions\ questions, for a total of \numpoints\ points and \numbonuspoints\ extra-credit points.

\noindent Unless otherwise stated, each question requires \emph{clear}, \emph{logically correct}, and \emph{sufficient} justification to convince the reader.

\noindent For bonus/extra-credit questions, we will provide very limited guidance in office hours and on Piazza, and we do not guarantee anything about the difficulty of these questions.
 
\noindent We strongly encourage you to typeset your solutions in \LaTeX.

\noindent If you collaborated with someone, you must state their name(s). You must \emph{write your own solution} for all problems and \emph{may not use any other student’s write-up}.
}

\newcommand{\hint}[1]{
\emph{Hint}: #1
}

% exam class setup
\pointsinmargin
\pointpoints{pt}{pts}
\bonuspointpoints{EC pt}{EC pts}
\marginpointname{ \points}
\marginbonuspointname{ \bonuspoints}
